\documentclass{article}
\usepackage[utf8]{inputenc}
\usepackage[margin = 1in]{geometry}
\title{Covariance matrix for coefficients is positive definite}
\author{Uma Roy}
\date{\today}

\usepackage{natbib}
\usepackage[colorlinks]{hyperref}
\usepackage{graphicx}
\usepackage{algorithm}
\usepackage{algorithmicx}
\usepackage{algpseudocode}
\usepackage{amsmath, amssymb, amsthm}
\theoremstyle{definition}
\newtheorem{example}{Example}[section]
\newtheorem{definition}{Definition}
\newtheorem{theorem}{Theorem}
\newcommand{\ISS}{\text{ISS}}
\newcommand{\Cov}{\mathrm{Cov}}
\newcommand{\R}{\mathbb{R}}
\newcommand{\E}{\mathbb{E}}
\newcommand{\X}{\vec{X}}
\newcommand{\neigh}[1]{\text{neigh}(#1)}
\newcommand{\I}{\mathcal{I}}
\begin{document}

\maketitle

Let $\Sigma$ be the covariance matrix of a Gaussian graphical model on $p$ nodes (i.e. $\Sigma \in \mathbb{R}^{p \times p}$) and let $N_1, N_2, \ldots, N_S$ be subsets of $[p]$ such that for all subsets $N_t$, they are of the form $N_t = \{ k, l, \neigh{k}, r_t \}$, where $r_t$ is a random vertex $r_t \not \in \{ k, l, \neigh{k} \}$ that satisfies the following property: \emph{There exists at least 1 path from $l$ to $r$ not going through $k$ or $\neigh{k}$ (i.e. $k$ and its neighbors are not vertex separators for $l$ and $r$)}.

Given a matrix $M$, we let $M_{N_t, N_q}$ denote the submatrix obtained by taking all rows corresponding to nodes in $N_t$ and all columns in $N_q$. We also let $S$ denote the empirical covariance matrix of samples drawn from our Gaussian graphical model. 

Let $\X$ be the vector where the $t$-th component $X_t = (S_{N_t, N_t})^{-1}_{k,l}$, where we abuse notation a bit by letting the entry $(k,l)$ of this matrix correspond to the position of $(k,l)$ in $N_t$ (as opposed to the position of $(k,l)$ in $[p]$, as would be normal). If we let $\sigma^{kk} := (\Sigma)^{-1}_{kk}$, then we have 

$$X_t = -\rho^{k,l}_{ \{ \neigh{k}, r_t \}} \cdot \sqrt{\sigma^{kk} \sigma^{ll}},$$ 

which is the negative partial covariance between nodes $k, l$ when conditioning on $\neigh{k}, r_t$. As long as $l \not \in \neigh{k}$, we have $\E[\rho^{k,l}_{ \{ \neigh{k}, r_t \}}] = 0$ (and in fact $\rho^{k,l}_{ \{ \neigh{k}, r_t \}}$ is normally distributed around $0$). 

$\Cov(\X)$ is an important quantity because it reveals how the coefficient of $k$ is correlated across the different regressions of node $k$ on subsets $N_t$. 
\begin{theorem}

Having defined $\X$ as above, we prove that $\Cov(\X)$ is positive definite (full-rank). Ideally, we would like control of the minimal eigenvalue of $\Cov(\X)$.

\end{theorem}


\begin{proof}[Short Version of Proof]
If we let $G = \ISS(\Sigma)$ and $I_t = \ISS(\Sigma_{N_t, N_t}^{-1})$, then we have $\Cov(\X) = \I G \I^T$, where $\I \in \mathbb{R}^{S \times d'}$, where $S$ is the number of different subsets we're considering and $d'$ is the dimension of $\ISS(\Sigma)$---so $\I$ has rows indexed by the different subsets and the columns indexed by the entries of $\Sigma$ (which we denote as tuples). In particular, we have the following formula for the entries of $\I$:

\begin{equation}
\I_{t, (u,v)} =
\begin{cases}
    (I_t)_{(k,l), (u,v)},& \text{if } (u,v) \in N_t \times N_t \\
    0,              & \text{otherwise.}
\end{cases}
\end{equation}


We know that $G$ is full-rank because it is an Isserlis matrix. Thus to show $\Cov(\X)$ is positive definite, it suffices to show that $v^T \I \neq \vec{0}$ for $v \neq \vec{0}$, i.e. $\I$ has full row-rank. To show that $\I$ has full row-rank, we show that for each row of $\I$, there is at least 1 entry that is non-zero \emph{only} in that row, which implies the rows are linearly independent. 

In particular for each $t$, consider the entry $\I_{t, (k, r_t)}$. By the formula for Isserlis matrices, we have that $\I_{t, (k,r_t)} = (I_t)_{(k,l), (k, r_t)} = \sigma_{N_t}^{kk}\sigma_{N_t}^{lr_t} + \sigma_{N_t}^{kr_t}\sigma_{N_t}^{lk} = \sigma_{N_t}^{kk}\sigma_{N_t}^{lr_t}$, since $\sigma_{N_t}^{lk} = 0$, since all paths from $k$ to $l$ must go through $\neigh{k} \in N_t$. However, since we imposed the condition that there is a path from $l$ to $r_t$ that does not go through $k$ or $\neigh{k}$, we have that $\sigma_{N_t}^{lr_t} \neq 0$ and since $\sigma_{N_t}^{kk} \neq 0$, we have that $\I_{t, (k, r_t)} \neq 0$. But consider the same entry in any other row, i.e. $\I_{q, (k, r_t)}$, for $q \neq t$. Then we have that since $r_t \not \in N_q$, this entry is necessarily $=0$. Thus we have shown each row of $\I$ has at least 1 entry that is non-zero in \emph{only} that row, proving that it has full row-rank, which completes the proof.


\end{proof}

\begin{proof}[Full proof]

By Taylor expansion, we have

$$ \left( S_{N_t, N_t} \right)^{-1}  = \left( \Sigma_{N_t N_t}\right)^{-1} + \ISS \left( \left( \Sigma_{N_t N_t}\right)^{-1} \right) \left(S_{N_tN_t} - \Sigma_{N_t N_t} \right).$$

Define $D_t$ such that $S_{N_t N_t} = D_t \cdot S$ (i.e. $D_t$ selects the relevant indices from $[p] \in N_t$). We note that we are treating $S_{N_t N_t}$ as a vector $\in \mathbb{R}^{d}$ and $S \in \mathbb{R}^{d'}$ as a vector, where $d = \frac{|N_t|(|N_t| + 1)}{2}$ and $d' = \frac{p(p+1)}{2}$. Then $D_t \in R^{d \times d'}$. 

We also know that $$\lim_{n \rightarrow \infty} (S - \Sigma) \rightarrow \mathcal{N}(0, G),$$ where $G \in \mathbb{R}^{d' \times d'}$ and $G = \ISS(\Sigma)$.

We note that 
$$ \mathbb{E} \left[ \left( S_{N_t, N_t} \right)^{-1} \right] =   \left( \Sigma_{N_t N_t}\right)^{-1},$$

since $\mathbb{E}[S - \Sigma] = \vec{0}$. 

Then to compute $\Cov(X_t, X_q)$, we must compute the following: 

$$\mathbb{E} \left[  \left( \left( S_{N_t, N_t} \right)^{-1} - \left( \Sigma_{N_t, N_t}\right)^{-1} \right) \left( \left( S_{N_q, N_q} \right)^{-1} - \left( \Sigma_{N_q, N_q}\right)^{-1} \right)^T \right]$$

$$ = \mathbb{E} \left[ \left( \ISS \left( \left( \Sigma_{N_t N_t}\right)^{-1} \right) \left(S_{N_tN_t} - \Sigma_{N_t N_t} \right) \right) \left( \ISS \left( \left( \Sigma_{N_q N_q}\right)^{-1} \right) \left(S_{N_qN_q} - \Sigma_{N_q N_q} \right) \right)^T \right]$$

Let $I_t = \ISS \left( \left( \Sigma_{N_t N_t}\right)^{-1} \right)$ and $I_q$ similarly.

$$ = \mathbb{E} \left[ (I_t \cdot D_t (S - \Sigma)) (I_q \cdot D_q (S-\Sigma))^T \right]$$

$$ = \mathbb{E} \left[ I_t \cdot D_t (S - \Sigma)(S - \Sigma)^T D_q^T I_q^T \right]$$

$$ = I_t \cdot D_t \cdot G \cdot D_q^T \cdot I_q^T.$$


Thus we can compute the $(t,q)$-th entry of $\Cov(\vec{X})$ as follows: 

$$\Cov(\vec{X})_{t, q} = \left( I_t \cdot D_t \cdot G \cdot D_q^T \cdot I_q^T \right)_{(k,l), (k,l)},$$

where we again abuse notation and take the entry of the resulting matrix that corresponds position of entry $(k,l)$ in $ \left( S_{N_t, N_t} \right)$ (when we treat it as a vector $\in \R^d$) and similarly for the position of $(k,l)$ in $\left( S_{N_q, N_q} \right)$.

Because of the matrix identity: $\left( ABC^T \right)_{i,j} = A_i B (C_j)^T$, where $A_i$ is the $i$-th row of $A$, if we let $I_t' := I_t \cdot D_t$, then we can rewrite 

$$\Cov(\vec{X})_{t, q} = \left( I_t \cdot D_t \cdot G \cdot D_q^T \cdot I_q^T \right)_{(k,l), (k,l)} = (I_t')_{(k,l)} G \left( (I_q')_{(k,l)} \right)^T,$$

where by yet another abuse of notation, $(I_t')_{(k,l)}$ denotes the row of $I_t'$ that corresponds to the pair $(k,l)$ in $S_{N_t, N_t}$ (regarded as a vector $\in \R^d$). 

If we let $\I \in \R^{S, d'}$ (where $S$ is the number of susbets we're considering) be the matrix where each row $\I_t = (I_t')_{(k,l)}$, again using the same matrix identity as before, we have 

$$\Cov(\X) = \I G \I^T.$$

{\color{blue} We know that $G$ is full-rank because it is an Isserlis matrix. } Thus to show $\Cov(\X)$ is positive definite, it suffices to show that $v^T \I \neq \vec{0}$ for $v \neq \vec{0}$, i.e. $\I$ has full row-rank. 

To show that $\I$ has full row-rank, we show that the rows of $\I$ are all linearly independent. In particular, we show that for each row of $\I$, there is at least 1 entry that is non-zero \emph{only} in that row, which implies the rows are linearly independent. 

Recall that the $t$-th row of $\I$ is: $(I_t \cdot D_t)_{(k,l)} = (I_t)_{(k,l)} \cdot D_t$. Effectively $D_t$ is transforming entries of $I_t$ to the corresponding indexing used for all nodes (since $I_t$ is indexed by only the nodes in $N_t$). In particular, if we let $\I_{t, (u,v)}$ denote the entry in row $t$ of $\I$ that corresponds to the pair of nodes $(u,v)$, then 

\begin{equation}
\I_{t, (u,v)} =
\begin{cases}
    (I_t)_{(k,l), (u,v)},& \text{if } (u,v) \in N_t \times N_t \\
    0,              & \text{otherwise.}
\end{cases}
\end{equation}

In particular for each $t$, consider the entry $\I_{t, (k, r_t)}$. By the formula for Isserlis matrices, we have that $\I_{t, (k,r_t)} = (I_t)_{(k,l), (k, r_t)} = \sigma_{N_t}^{kk}\sigma_{N_t}^{lr_t} + \sigma_{N_t}^{kr_t}\sigma_{N_t}^{lk} = \sigma_{N_t}^{kk}\sigma_{N_t}^{lr_t}$, since $\sigma_{N_t}^{lk} = 0$, since all paths from $k$ to $l$ must go through $\neigh{k} \in N_t$. However, since we imposed the condition that there is a path from $l$ to $r_t$ that does not go through $k$ or $\neigh{k}$, we have that $\sigma_{N_t}^{lr_t} \neq 0$ and since $\sigma_{N_t}^{kk} \neq 0$, we have that $\I_{t, (k, r_t)} \neq 0$. But consider the same entry in any other row, i.e. $\I_{q, (k, r_t)}$, for $q \neq t$. Then we have that since $r_t \not \in N_q$, this entry is necessarily $=0$. Thus we have shown each row of $\I$ has at least 1 entry that is non-zero in \emph{only} that row, proving that it has full row-rank, which completes the proof.

\end{proof}


\end{document}
